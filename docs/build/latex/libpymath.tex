%% Generated by Sphinx.
\def\sphinxdocclass{report}
\documentclass[letterpaper,10pt,english]{sphinxmanual}
\ifdefined\pdfpxdimen
   \let\sphinxpxdimen\pdfpxdimen\else\newdimen\sphinxpxdimen
\fi \sphinxpxdimen=.75bp\relax

\PassOptionsToPackage{warn}{textcomp}
\usepackage[utf8]{inputenc}
\ifdefined\DeclareUnicodeCharacter
% support both utf8 and utf8x syntaxes
  \ifdefined\DeclareUnicodeCharacterAsOptional
    \def\sphinxDUC#1{\DeclareUnicodeCharacter{"#1}}
  \else
    \let\sphinxDUC\DeclareUnicodeCharacter
  \fi
  \sphinxDUC{00A0}{\nobreakspace}
  \sphinxDUC{2500}{\sphinxunichar{2500}}
  \sphinxDUC{2502}{\sphinxunichar{2502}}
  \sphinxDUC{2514}{\sphinxunichar{2514}}
  \sphinxDUC{251C}{\sphinxunichar{251C}}
  \sphinxDUC{2572}{\textbackslash}
\fi
\usepackage{cmap}
\usepackage[T1]{fontenc}
\usepackage{amsmath,amssymb,amstext}
\usepackage{babel}



\usepackage{times}
\expandafter\ifx\csname T@LGR\endcsname\relax
\else
% LGR was declared as font encoding
  \substitutefont{LGR}{\rmdefault}{cmr}
  \substitutefont{LGR}{\sfdefault}{cmss}
  \substitutefont{LGR}{\ttdefault}{cmtt}
\fi
\expandafter\ifx\csname T@X2\endcsname\relax
  \expandafter\ifx\csname T@T2A\endcsname\relax
  \else
  % T2A was declared as font encoding
    \substitutefont{T2A}{\rmdefault}{cmr}
    \substitutefont{T2A}{\sfdefault}{cmss}
    \substitutefont{T2A}{\ttdefault}{cmtt}
  \fi
\else
% X2 was declared as font encoding
  \substitutefont{X2}{\rmdefault}{cmr}
  \substitutefont{X2}{\sfdefault}{cmss}
  \substitutefont{X2}{\ttdefault}{cmtt}
\fi


\usepackage[Bjarne]{fncychap}
\usepackage{sphinx}

\fvset{fontsize=\small}
\usepackage{geometry}


% Include hyperref last.
\usepackage{hyperref}
% Fix anchor placement for figures with captions.
\usepackage{hypcap}% it must be loaded after hyperref.
% Set up styles of URL: it should be placed after hyperref.
\urlstyle{same}


\usepackage{sphinxmessages}
\setcounter{tocdepth}{1}



\title{libpymath}
\date{Oct 19, 2020}
\release{0.6.0}
\author{Toby Davis}
\newcommand{\sphinxlogo}{\vbox{}}
\renewcommand{\releasename}{Release}
\makeindex
\begin{document}

\pagestyle{empty}
\sphinxmaketitle
\pagestyle{plain}
\sphinxtableofcontents
\pagestyle{normal}
\phantomsection\label{\detokenize{index::doc}}


Libpymath \sphinxhyphen{} a fast, general purpose Python math library


\chapter{The Aim}
\label{\detokenize{index:the-aim}}
The aim of libpymath is to provide a simple and intuitive inerface to complex mathematical routines, while maintaining a fast and efficient operation. The module will allow you to create powerful matrices, progress bars, neural networks and more with only a few lines of code.


\section{Licencing}
\label{\detokenize{index:licencing}}
Libpymath is covered under an MIT license, so you can use all of the code as is for private, open source or commercial projects, as long as the copyright notice is retained at the top of any files used.


\chapter{Installing the Package}
\label{\detokenize{index:installing-the-package}}
To install the package, simply open your CLI of choice and run \sphinxcode{\sphinxupquote{pip install libpymath}} to install the package. Hopefully there will be wheels provided for your operating system and Python version, however none exist the code must be built from source. This only takes a few seconds, however you will need a C or C++ compiler installed on your system.

If you would like to upgrade to the lastest version of libpymath, you can run \sphinxcode{\sphinxupquote{pip install libpymath \sphinxhyphen{}\sphinxhyphen{}upgrade}} to upgrade an existing version installed on your system.

Another option for more experienced users is to clone the repository using the \sphinxcode{\sphinxupquote{git clone https://github.com/Pencilcaseman/LibPyMath.git}} command in a directory of your choice. This will install the source code for you, allowing you to edit the code, add features, fix potential bugs (hopefully you won’t find any) or simply see how it works.


\chapter{Using the Package}
\label{\detokenize{index:using-the-package}}
With libpymath installed, you can open a Python environment, such as IDLE, and start to write your first piece of code.

To import the module, add this to the top of your file:

\begin{sphinxVerbatim}[commandchars=\\\{\}]
\PYG{k+kn}{import} \PYG{n+nn}{libpymath}
\end{sphinxVerbatim}

Libpymath is commonly imported as \sphinxcode{\sphinxupquote{lpm}}, which is shorter and easier to type. This change can be made easily:

\begin{sphinxVerbatim}[commandchars=\\\{\}]
\PYG{k+kn}{import} \PYG{n+nn}{libpymath} \PYG{k}{as} \PYG{n+nn}{lpm}
\end{sphinxVerbatim}

The libpymath import allows you to access all of the features of libpymath, such as the matrix library or the progress bar library. For information on how to use these features, see below.


\chapter{Contents}
\label{\detokenize{index:contents}}

\section{libpymath}
\label{\detokenize{modules:libpymath}}\label{\detokenize{modules::doc}}

\subsection{libpymath package}
\label{\detokenize{libpymath:libpymath-package}}\label{\detokenize{libpymath::doc}}

\subsubsection{Subpackages}
\label{\detokenize{libpymath:subpackages}}

\paragraph{libpymath.matrix package}
\label{\detokenize{libpymath.matrix:libpymath-matrix-package}}\label{\detokenize{libpymath.matrix::doc}}

\subparagraph{Submodules}
\label{\detokenize{libpymath.matrix:submodules}}

\subparagraph{libpymath.matrix.matrix module}
\label{\detokenize{libpymath.matrix:module-libpymath.matrix.matrix}}\label{\detokenize{libpymath.matrix:libpymath-matrix-matrix-module}}\index{module@\spxentry{module}!libpymath.matrix.matrix@\spxentry{libpymath.matrix.matrix}}\index{libpymath.matrix.matrix@\spxentry{libpymath.matrix.matrix}!module@\spxentry{module}}\index{Matrix (class in libpymath.matrix.matrix)@\spxentry{Matrix}\spxextra{class in libpymath.matrix.matrix}}

\begin{fulllineitems}
\phantomsection\label{\detokenize{libpymath.matrix:libpymath.matrix.matrix.Matrix}}\pysiglinewithargsret{\sphinxbfcode{\sphinxupquote{class }}\sphinxcode{\sphinxupquote{libpymath.matrix.matrix.}}\sphinxbfcode{\sphinxupquote{Matrix}}}{\emph{\DUrole{o}{*}\DUrole{n}{args}}, \emph{\DUrole{o}{**}\DUrole{n}{kwargs}}}{}
Bases: \sphinxcode{\sphinxupquote{object}}
\index{T() (libpymath.matrix.matrix.Matrix property)@\spxentry{T()}\spxextra{libpymath.matrix.matrix.Matrix property}}

\begin{fulllineitems}
\phantomsection\label{\detokenize{libpymath.matrix:libpymath.matrix.matrix.Matrix.T}}\pysigline{\sphinxbfcode{\sphinxupquote{property }}\sphinxbfcode{\sphinxupquote{T}}}
See Matrix.transposed()
\begin{quote}\begin{description}
\item[{Returns}] \leavevmode
Return the transpose of a matrix

\end{description}\end{quote}

\end{fulllineitems}

\index{cols() (libpymath.matrix.matrix.Matrix property)@\spxentry{cols()}\spxextra{libpymath.matrix.matrix.Matrix property}}

\begin{fulllineitems}
\phantomsection\label{\detokenize{libpymath.matrix:libpymath.matrix.matrix.Matrix.cols}}\pysigline{\sphinxbfcode{\sphinxupquote{property }}\sphinxbfcode{\sphinxupquote{cols}}}
The number of columns of the matrix
\begin{quote}\begin{description}
\item[{Type}] \leavevmode
return

\end{description}\end{quote}

\end{fulllineitems}

\index{copy() (libpymath.matrix.matrix.Matrix method)@\spxentry{copy()}\spxextra{libpymath.matrix.matrix.Matrix method}}

\begin{fulllineitems}
\phantomsection\label{\detokenize{libpymath.matrix:libpymath.matrix.matrix.Matrix.copy}}\pysiglinewithargsret{\sphinxbfcode{\sphinxupquote{copy}}}{}{}
Return an exact copy of a matrix
\begin{quote}\begin{description}
\item[{Returns}] \leavevmode
Copied matrix

\end{description}\end{quote}

\end{fulllineitems}

\index{dot() (libpymath.matrix.matrix.Matrix method)@\spxentry{dot()}\spxextra{libpymath.matrix.matrix.Matrix method}}

\begin{fulllineitems}
\phantomsection\label{\detokenize{libpymath.matrix:libpymath.matrix.matrix.Matrix.dot}}\pysiglinewithargsret{\sphinxbfcode{\sphinxupquote{dot}}}{\emph{\DUrole{n}{other}}}{}
Compute the matrix\sphinxhyphen{}matrix product with another matrix
\begin{quote}\begin{description}
\item[{Parameters}] \leavevmode
\sphinxstyleliteralstrong{\sphinxupquote{other}} \textendash{} Matrix to compute matrix product with

\item[{Returns}] \leavevmode
Result of matrix product calculation

\end{description}\end{quote}

\end{fulllineitems}

\index{dtype() (libpymath.matrix.matrix.Matrix property)@\spxentry{dtype()}\spxextra{libpymath.matrix.matrix.Matrix property}}

\begin{fulllineitems}
\phantomsection\label{\detokenize{libpymath.matrix:libpymath.matrix.matrix.Matrix.dtype}}\pysigline{\sphinxbfcode{\sphinxupquote{property }}\sphinxbfcode{\sphinxupquote{dtype}}}
The datatype of the matrix
\begin{quote}\begin{description}
\item[{Type}] \leavevmode
return

\end{description}\end{quote}

\end{fulllineitems}

\index{fill() (libpymath.matrix.matrix.Matrix method)@\spxentry{fill()}\spxextra{libpymath.matrix.matrix.Matrix method}}

\begin{fulllineitems}
\phantomsection\label{\detokenize{libpymath.matrix:libpymath.matrix.matrix.Matrix.fill}}\pysiglinewithargsret{\sphinxbfcode{\sphinxupquote{fill}}}{\emph{\DUrole{n}{fillType}}, \emph{\DUrole{o}{*}\DUrole{n}{args}}}{}
Fill a matrix. Valid fills are:

SCALAR      \sphinxhyphen{}\textgreater{} Fill with a single scalar value
ASCENDING   \sphinxhyphen{}\textgreater{} Fill a matrix with ascending values, starting with 0
DESCENDING  \sphinxhyphen{}\textgreater{} Fill a matrix with descending values, starting from (rows * cols) \sphinxhyphen{} 1
RANDOM      \sphinxhyphen{}\textgreater{} Fill a matrix with random values between a given range (defaults to {[}\sphinxhyphen{}1, 1|)
\begin{quote}\begin{description}
\item[{Parameters}] \leavevmode\begin{itemize}
\item {} 
\sphinxstyleliteralstrong{\sphinxupquote{fillType}} \textendash{} Method to use when filling the matrix

\item {} 
\sphinxstyleliteralstrong{\sphinxupquote{args}} \textendash{} Some fill methods accept parameters

\end{itemize}

\item[{Returns}] \leavevmode
None

\end{description}\end{quote}

\end{fulllineitems}

\index{fillAscending() (libpymath.matrix.matrix.Matrix method)@\spxentry{fillAscending()}\spxextra{libpymath.matrix.matrix.Matrix method}}

\begin{fulllineitems}
\phantomsection\label{\detokenize{libpymath.matrix:libpymath.matrix.matrix.Matrix.fillAscending}}\pysiglinewithargsret{\sphinxbfcode{\sphinxupquote{fillAscending}}}{}{}
See Matrix.map(ASCENDING)
\begin{quote}\begin{description}
\item[{Returns}] \leavevmode
None

\end{description}\end{quote}

\end{fulllineitems}

\index{fillDescending() (libpymath.matrix.matrix.Matrix method)@\spxentry{fillDescending()}\spxextra{libpymath.matrix.matrix.Matrix method}}

\begin{fulllineitems}
\phantomsection\label{\detokenize{libpymath.matrix:libpymath.matrix.matrix.Matrix.fillDescending}}\pysiglinewithargsret{\sphinxbfcode{\sphinxupquote{fillDescending}}}{}{}
See Matrix.map(DESCENDING)
\begin{quote}\begin{description}
\item[{Returns}] \leavevmode
None

\end{description}\end{quote}

\end{fulllineitems}

\index{fillRandom() (libpymath.matrix.matrix.Matrix method)@\spxentry{fillRandom()}\spxextra{libpymath.matrix.matrix.Matrix method}}

\begin{fulllineitems}
\phantomsection\label{\detokenize{libpymath.matrix:libpymath.matrix.matrix.Matrix.fillRandom}}\pysiglinewithargsret{\sphinxbfcode{\sphinxupquote{fillRandom}}}{\emph{\DUrole{n}{\_min}\DUrole{o}{=}\DUrole{default_value}{None}}, \emph{\DUrole{n}{\_max}\DUrole{o}{=}\DUrole{default_value}{None}}}{}
See Matrix.map(RANDOM)
\begin{quote}\begin{description}
\item[{Returns}] \leavevmode
None

\end{description}\end{quote}

\end{fulllineitems}

\index{fillScalar() (libpymath.matrix.matrix.Matrix method)@\spxentry{fillScalar()}\spxextra{libpymath.matrix.matrix.Matrix method}}

\begin{fulllineitems}
\phantomsection\label{\detokenize{libpymath.matrix:libpymath.matrix.matrix.Matrix.fillScalar}}\pysiglinewithargsret{\sphinxbfcode{\sphinxupquote{fillScalar}}}{\emph{\DUrole{n}{x}}}{}
See Matrix.map(SCALAR)
\begin{quote}\begin{description}
\item[{Parameters}] \leavevmode
\sphinxstyleliteralstrong{\sphinxupquote{x}} \textendash{} Scalar value to fill with

\item[{Returns}] \leavevmode
None

\end{description}\end{quote}

\end{fulllineitems}

\index{map() (libpymath.matrix.matrix.Matrix method)@\spxentry{map()}\spxextra{libpymath.matrix.matrix.Matrix method}}

\begin{fulllineitems}
\phantomsection\label{\detokenize{libpymath.matrix:libpymath.matrix.matrix.Matrix.map}}\pysiglinewithargsret{\sphinxbfcode{\sphinxupquote{map}}}{\emph{\DUrole{n}{mapType}}}{}
Apply a function to every element of the matrix.

Valid functions are:
SIGMOID
TANH
RELU
LEAKY\_RELU

D\_SIGMOID
D\_TANH
D\_RELU
D\_LEAKY\_RELU
\begin{quote}\begin{description}
\item[{Parameters}] \leavevmode
\sphinxstyleliteralstrong{\sphinxupquote{mapType}} \textendash{} Function to map with

\item[{Returns}] \leavevmode
None

\end{description}\end{quote}

\end{fulllineitems}

\index{mapped() (libpymath.matrix.matrix.Matrix method)@\spxentry{mapped()}\spxextra{libpymath.matrix.matrix.Matrix method}}

\begin{fulllineitems}
\phantomsection\label{\detokenize{libpymath.matrix:libpymath.matrix.matrix.Matrix.mapped}}\pysiglinewithargsret{\sphinxbfcode{\sphinxupquote{mapped}}}{\emph{\DUrole{n}{mapType}}}{}
See Matrix.map()
\begin{quote}\begin{description}
\item[{Parameters}] \leavevmode
\sphinxstyleliteralstrong{\sphinxupquote{mapType}} \textendash{} Function to map with

\item[{Returns}] \leavevmode
Mapped matrix

\end{description}\end{quote}

\end{fulllineitems}

\index{mean() (libpymath.matrix.matrix.Matrix method)@\spxentry{mean()}\spxextra{libpymath.matrix.matrix.Matrix method}}

\begin{fulllineitems}
\phantomsection\label{\detokenize{libpymath.matrix:libpymath.matrix.matrix.Matrix.mean}}\pysiglinewithargsret{\sphinxbfcode{\sphinxupquote{mean}}}{}{}
\end{fulllineitems}

\index{reshape() (libpymath.matrix.matrix.Matrix method)@\spxentry{reshape()}\spxextra{libpymath.matrix.matrix.Matrix method}}

\begin{fulllineitems}
\phantomsection\label{\detokenize{libpymath.matrix:libpymath.matrix.matrix.Matrix.reshape}}\pysiglinewithargsret{\sphinxbfcode{\sphinxupquote{reshape}}}{\emph{\DUrole{n}{nr}}, \emph{\DUrole{n}{nc}}}{}
Reshape a matrix by adjusting the rows and columns.
\begin{quote}\begin{description}
\item[{Parameters}] \leavevmode\begin{itemize}
\item {} 
\sphinxstyleliteralstrong{\sphinxupquote{nr}} \textendash{} New rows

\item {} 
\sphinxstyleliteralstrong{\sphinxupquote{nc}} \textendash{} New columns

\end{itemize}

\item[{Returns}] \leavevmode
None

\end{description}\end{quote}

\end{fulllineitems}

\index{reshaped() (libpymath.matrix.matrix.Matrix method)@\spxentry{reshaped()}\spxextra{libpymath.matrix.matrix.Matrix method}}

\begin{fulllineitems}
\phantomsection\label{\detokenize{libpymath.matrix:libpymath.matrix.matrix.Matrix.reshaped}}\pysiglinewithargsret{\sphinxbfcode{\sphinxupquote{reshaped}}}{\emph{\DUrole{n}{nr}}, \emph{\DUrole{n}{nc}}}{}
Reshape a matrix by adjusting the rows and columns and return the result.
\begin{quote}\begin{description}
\item[{Parameters}] \leavevmode\begin{itemize}
\item {} 
\sphinxstyleliteralstrong{\sphinxupquote{nr}} \textendash{} New rows

\item {} 
\sphinxstyleliteralstrong{\sphinxupquote{nc}} \textendash{} New columns

\end{itemize}

\item[{Returns}] \leavevmode
Reshaped matrix

\end{description}\end{quote}

\end{fulllineitems}

\index{rows() (libpymath.matrix.matrix.Matrix property)@\spxentry{rows()}\spxextra{libpymath.matrix.matrix.Matrix property}}

\begin{fulllineitems}
\phantomsection\label{\detokenize{libpymath.matrix:libpymath.matrix.matrix.Matrix.rows}}\pysigline{\sphinxbfcode{\sphinxupquote{property }}\sphinxbfcode{\sphinxupquote{rows}}}
The number of rows of the matrix
\begin{quote}\begin{description}
\item[{Type}] \leavevmode
return

\end{description}\end{quote}

\end{fulllineitems}

\index{shape() (libpymath.matrix.matrix.Matrix property)@\spxentry{shape()}\spxextra{libpymath.matrix.matrix.Matrix property}}

\begin{fulllineitems}
\phantomsection\label{\detokenize{libpymath.matrix:libpymath.matrix.matrix.Matrix.shape}}\pysigline{\sphinxbfcode{\sphinxupquote{property }}\sphinxbfcode{\sphinxupquote{shape}}}
The shape of the matrix in the form (rows, columns)
\begin{quote}\begin{description}
\item[{Type}] \leavevmode
return

\end{description}\end{quote}

\end{fulllineitems}

\index{sum() (libpymath.matrix.matrix.Matrix method)@\spxentry{sum()}\spxextra{libpymath.matrix.matrix.Matrix method}}

\begin{fulllineitems}
\phantomsection\label{\detokenize{libpymath.matrix:libpymath.matrix.matrix.Matrix.sum}}\pysiglinewithargsret{\sphinxbfcode{\sphinxupquote{sum}}}{}{}
\end{fulllineitems}

\index{threads() (libpymath.matrix.matrix.Matrix property)@\spxentry{threads()}\spxextra{libpymath.matrix.matrix.Matrix property}}

\begin{fulllineitems}
\phantomsection\label{\detokenize{libpymath.matrix:libpymath.matrix.matrix.Matrix.threads}}\pysigline{\sphinxbfcode{\sphinxupquote{property }}\sphinxbfcode{\sphinxupquote{threads}}}
The number of threads the matrix is using
\begin{quote}\begin{description}
\item[{Type}] \leavevmode
return

\end{description}\end{quote}

\end{fulllineitems}

\index{toList() (libpymath.matrix.matrix.Matrix method)@\spxentry{toList()}\spxextra{libpymath.matrix.matrix.Matrix method}}

\begin{fulllineitems}
\phantomsection\label{\detokenize{libpymath.matrix:libpymath.matrix.matrix.Matrix.toList}}\pysiglinewithargsret{\sphinxbfcode{\sphinxupquote{toList}}}{}{}
Convert a Matrix into a 2d Python list
\begin{quote}\begin{description}
\item[{Returns}] \leavevmode
2d Python list

\end{description}\end{quote}

\end{fulllineitems}

\index{transpose() (libpymath.matrix.matrix.Matrix method)@\spxentry{transpose()}\spxextra{libpymath.matrix.matrix.Matrix method}}

\begin{fulllineitems}
\phantomsection\label{\detokenize{libpymath.matrix:libpymath.matrix.matrix.Matrix.transpose}}\pysiglinewithargsret{\sphinxbfcode{\sphinxupquote{transpose}}}{}{}
Transpose a matrix inplace.
\begin{quote}\begin{description}
\item[{Returns}] \leavevmode
None

\end{description}\end{quote}

\end{fulllineitems}

\index{transposed() (libpymath.matrix.matrix.Matrix method)@\spxentry{transposed()}\spxextra{libpymath.matrix.matrix.Matrix method}}

\begin{fulllineitems}
\phantomsection\label{\detokenize{libpymath.matrix:libpymath.matrix.matrix.Matrix.transposed}}\pysiglinewithargsret{\sphinxbfcode{\sphinxupquote{transposed}}}{}{}
See Matrix.transpose()
\begin{quote}\begin{description}
\item[{Returns}] \leavevmode
Return the transpose of a matrix

\end{description}\end{quote}

\end{fulllineitems}


\end{fulllineitems}



\subparagraph{Module contents}
\label{\detokenize{libpymath.matrix:module-libpymath.matrix}}\label{\detokenize{libpymath.matrix:module-contents}}\index{module@\spxentry{module}!libpymath.matrix@\spxentry{libpymath.matrix}}\index{libpymath.matrix@\spxentry{libpymath.matrix}!module@\spxentry{module}}
Copyright 2020 Toby Davis

Permission is hereby granted, free of charge, to any person obtaining a copy of
this software and associated documentation files (the “Software”), to deal in
the Software without restriction, including without limitation the rights to
use, copy, modify, merge, publish, distribute, sublicense, and/or sell copies
of the Software, and to permit persons to whom the Software is furnished to do
so, subject to the following conditions:

The above copyright notice and this permission notice shall be included in all
copies or substantial portions of the Software.

THE SOFTWARE IS PROVIDED “AS IS”, WITHOUT WARRANTY OF ANY KIND, EXPRESS OR
IMPLIED, INCLUDING BUT NOT LIMITED TO THE WARRANTIES OF MERCHANTABILITY, FITNESS
FOR A PARTICULAR PURPOSE AND NONINFRINGEMENT. IN NO EVENT SHALL THE AUTHORS OR
COPYRIGHT HOLDERS BE LIABLE FOR ANY CLAIM, DAMAGES OR OTHER LIABILITY, WHETHER
IN AN ACTION OF CONTRACT, TORT OR OTHERWISE, ARISING FROM, OUT OF OR IN
CONNECTION WITH THE SOFTWARE OR THE USE OR OTHER DEALINGS IN THE SOFTWARE.


\paragraph{libpymath.progress package}
\label{\detokenize{libpymath.progress:libpymath-progress-package}}\label{\detokenize{libpymath.progress::doc}}

\subparagraph{Submodules}
\label{\detokenize{libpymath.progress:submodules}}

\subparagraph{libpymath.progress.progress module}
\label{\detokenize{libpymath.progress:module-libpymath.progress.progress}}\label{\detokenize{libpymath.progress:libpymath-progress-progress-module}}\index{module@\spxentry{module}!libpymath.progress.progress@\spxentry{libpymath.progress.progress}}\index{libpymath.progress.progress@\spxentry{libpymath.progress.progress}!module@\spxentry{module}}\index{Progress (class in libpymath.progress.progress)@\spxentry{Progress}\spxextra{class in libpymath.progress.progress}}

\begin{fulllineitems}
\phantomsection\label{\detokenize{libpymath.progress:libpymath.progress.progress.Progress}}\pysiglinewithargsret{\sphinxbfcode{\sphinxupquote{class }}\sphinxcode{\sphinxupquote{libpymath.progress.progress.}}\sphinxbfcode{\sphinxupquote{Progress}}}{\emph{\DUrole{n}{iterable}\DUrole{o}{=}\DUrole{default_value}{None}}, \emph{\DUrole{n}{message}\DUrole{o}{=}\DUrole{default_value}{None}}, \emph{\DUrole{n}{erase}\DUrole{o}{=}\DUrole{default_value}{False}}, \emph{\DUrole{n}{start}\DUrole{o}{=}\DUrole{default_value}{None}}, \emph{\DUrole{n}{end}\DUrole{o}{=}\DUrole{default_value}{None}}, \emph{\DUrole{n}{step}\DUrole{o}{=}\DUrole{default_value}{None}}, \emph{\DUrole{n}{smoothness}\DUrole{o}{=}\DUrole{default_value}{0.85}}}{}
Bases: \sphinxcode{\sphinxupquote{object}}
\index{generateBar() (libpymath.progress.progress.Progress static method)@\spxentry{generateBar()}\spxextra{libpymath.progress.progress.Progress static method}}

\begin{fulllineitems}
\phantomsection\label{\detokenize{libpymath.progress:libpymath.progress.progress.Progress.generateBar}}\pysiglinewithargsret{\sphinxbfcode{\sphinxupquote{static }}\sphinxbfcode{\sphinxupquote{generateBar}}}{\emph{\DUrole{n}{fill}}, \emph{\DUrole{n}{length}}, \emph{\DUrole{n}{fillChar}\DUrole{o}{=}\DUrole{default_value}{\textquotesingle{}█\textquotesingle{}}}, \emph{\DUrole{n}{emptyChar}\DUrole{o}{=}\DUrole{default_value}{\textquotesingle{}█\textquotesingle{}}}}{}
\end{fulllineitems}

\index{reset() (libpymath.progress.progress.Progress method)@\spxentry{reset()}\spxextra{libpymath.progress.progress.Progress method}}

\begin{fulllineitems}
\phantomsection\label{\detokenize{libpymath.progress:libpymath.progress.progress.Progress.reset}}\pysiglinewithargsret{\sphinxbfcode{\sphinxupquote{reset}}}{}{}
\end{fulllineitems}

\index{update() (libpymath.progress.progress.Progress method)@\spxentry{update()}\spxextra{libpymath.progress.progress.Progress method}}

\begin{fulllineitems}
\phantomsection\label{\detokenize{libpymath.progress:libpymath.progress.progress.Progress.update}}\pysiglinewithargsret{\sphinxbfcode{\sphinxupquote{update}}}{}{}
\end{fulllineitems}


\end{fulllineitems}

\index{frange (class in libpymath.progress.progress)@\spxentry{frange}\spxextra{class in libpymath.progress.progress}}

\begin{fulllineitems}
\phantomsection\label{\detokenize{libpymath.progress:libpymath.progress.progress.frange}}\pysiglinewithargsret{\sphinxbfcode{\sphinxupquote{class }}\sphinxcode{\sphinxupquote{libpymath.progress.progress.}}\sphinxbfcode{\sphinxupquote{frange}}}{\emph{\DUrole{n}{start}}, \emph{\DUrole{n}{end}\DUrole{o}{=}\DUrole{default_value}{None}}, \emph{\DUrole{n}{step}\DUrole{o}{=}\DUrole{default_value}{None}}}{}
Bases: \sphinxcode{\sphinxupquote{object}}

\end{fulllineitems}



\subparagraph{Module contents}
\label{\detokenize{libpymath.progress:module-libpymath.progress}}\label{\detokenize{libpymath.progress:module-contents}}\index{module@\spxentry{module}!libpymath.progress@\spxentry{libpymath.progress}}\index{libpymath.progress@\spxentry{libpymath.progress}!module@\spxentry{module}}

\paragraph{libpymath.src package}
\label{\detokenize{libpymath.src:libpymath-src-package}}\label{\detokenize{libpymath.src::doc}}

\subparagraph{Subpackages}
\label{\detokenize{libpymath.src:subpackages}}

\subparagraph{libpymath.src.error package}
\label{\detokenize{libpymath.src.error:libpymath-src-error-package}}\label{\detokenize{libpymath.src.error::doc}}

\subparagraph{Module contents}
\label{\detokenize{libpymath.src.error:module-libpymath.src.error}}\label{\detokenize{libpymath.src.error:module-contents}}\index{module@\spxentry{module}!libpymath.src.error@\spxentry{libpymath.src.error}}\index{libpymath.src.error@\spxentry{libpymath.src.error}!module@\spxentry{module}}

\subparagraph{libpymath.src.matrix package}
\label{\detokenize{libpymath.src.matrix:libpymath-src-matrix-package}}\label{\detokenize{libpymath.src.matrix::doc}}

\subparagraph{Module contents}
\label{\detokenize{libpymath.src.matrix:module-libpymath.src.matrix}}\label{\detokenize{libpymath.src.matrix:module-contents}}\index{module@\spxentry{module}!libpymath.src.matrix@\spxentry{libpymath.src.matrix}}\index{libpymath.src.matrix@\spxentry{libpymath.src.matrix}!module@\spxentry{module}}

\subparagraph{Module contents}
\label{\detokenize{libpymath.src:module-libpymath.src}}\label{\detokenize{libpymath.src:module-contents}}\index{module@\spxentry{module}!libpymath.src@\spxentry{libpymath.src}}\index{libpymath.src@\spxentry{libpymath.src}!module@\spxentry{module}}

\subsubsection{Module contents}
\label{\detokenize{libpymath:module-libpymath}}\label{\detokenize{libpymath:module-contents}}\index{module@\spxentry{module}!libpymath@\spxentry{libpymath}}\index{libpymath@\spxentry{libpymath}!module@\spxentry{module}}
Copyright 2020 Toby Davis

Permission is hereby granted, free of charge, to any person obtaining a copy of
this software and associated documentation files (the “Software”), to deal in
the Software without restriction, including without limitation the rights to
use, copy, modify, merge, publish, distribute, sublicense, and/or sell copies
of the Software, and to permit persons to whom the Software is furnished to do
so, subject to the following conditions:

The above copyright notice and this permission notice shall be included in all
copies or substantial portions of the Software.

THE SOFTWARE IS PROVIDED “AS IS”, WITHOUT WARRANTY OF ANY KIND, EXPRESS OR
IMPLIED, INCLUDING BUT NOT LIMITED TO THE WARRANTIES OF MERCHANTABILITY, FITNESS
FOR A PARTICULAR PURPOSE AND NONINFRINGEMENT. IN NO EVENT SHALL THE AUTHORS OR
COPYRIGHT HOLDERS BE LIABLE FOR ANY CLAIM, DAMAGES OR OTHER LIABILITY, WHETHER
IN AN ACTION OF CONTRACT, TORT OR OTHERWISE, ARISING FROM, OUT OF OR IN
CONNECTION WITH THE SOFTWARE OR THE USE OR OTHER DEALINGS IN THE SOFTWARE.


\chapter{Indices and tables}
\label{\detokenize{index:indices-and-tables}}\begin{itemize}
\item {} 
\DUrole{xref,std,std-ref}{search}

\end{itemize}


\renewcommand{\indexname}{Python Module Index}
\begin{sphinxtheindex}
\let\bigletter\sphinxstyleindexlettergroup
\bigletter{l}
\item\relax\sphinxstyleindexentry{libpymath}\sphinxstyleindexpageref{libpymath:\detokenize{module-libpymath}}
\item\relax\sphinxstyleindexentry{libpymath.matrix}\sphinxstyleindexpageref{libpymath.matrix:\detokenize{module-libpymath.matrix}}
\item\relax\sphinxstyleindexentry{libpymath.matrix.matrix}\sphinxstyleindexpageref{libpymath.matrix:\detokenize{module-libpymath.matrix.matrix}}
\item\relax\sphinxstyleindexentry{libpymath.progress}\sphinxstyleindexpageref{libpymath.progress:\detokenize{module-libpymath.progress}}
\item\relax\sphinxstyleindexentry{libpymath.progress.progress}\sphinxstyleindexpageref{libpymath.progress:\detokenize{module-libpymath.progress.progress}}
\item\relax\sphinxstyleindexentry{libpymath.src}\sphinxstyleindexpageref{libpymath.src:\detokenize{module-libpymath.src}}
\item\relax\sphinxstyleindexentry{libpymath.src.error}\sphinxstyleindexpageref{libpymath.src.error:\detokenize{module-libpymath.src.error}}
\item\relax\sphinxstyleindexentry{libpymath.src.matrix}\sphinxstyleindexpageref{libpymath.src.matrix:\detokenize{module-libpymath.src.matrix}}
\end{sphinxtheindex}

\renewcommand{\indexname}{Index}
\printindex
\end{document}